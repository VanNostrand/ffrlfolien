\usepackage{ngerman,graphicx,framed}	%deutsche Sprache, Bilder, Rahmen
\usepackage[utf8]{inputenc}		%direkt deutsche Umlaute und EUR-Zeichen eingeben
\usepackage{lmodern}
\usepackage{textcomp}			%das EUR-Zeichen für OT und T1
\usepackage{color}			%Farbpaket
\usepackage{amsmath}			%Mathepaket - was sonst? =)
\usepackage{amsthm}			%theorem-Paket für Beweise, z.B. begin{proof} oder \qed
\usepackage{amsfonts}			%Schrift um z.B. Mathesymbole wie Folgerungspfeile zu schreiben
\usepackage{amssymb}			%mehr Symbole - yes! =)
\usepackage{stmaryrd}			%\lightning für Widerspruch
\usepackage{flafter}			%Gleitobjekte besser positionieren
\usepackage{placeins}			%it defines a \FloatBarrier command beyond which floats may not pass.
					%A package option allows you to declare that floats may not pass a
					%\section command, but you can place \FloatBarriers wherever you choose.
					%URL: http://www.tex.ac.uk/cgi-bin/texfaq2html?label=floats
\usepackage{mathrsfs}			%Schreibschriftbuchstaben für Mathemodus: \mathscr{Buchstabe}
\usepackage[mathcal]{euscript}		%Kaligraphiebuchstaben für Mathemodus: \mathcal{Buchstabe}
\usepackage{enumerate}			%Aufzählungen mit römischen Zahlen usw.
\usepackage[ngerman]{babel}
\usepackage{algorithmic}		%Quellcode einbauen
\usepackage{longtable}
\usepackage{booktabs}			%für besser aussehende Tabellen als das Standardzeug. 
\usepackage{cite}
\usepackage{listing}
% URL: http://www.ctan.org/tex-archive/macros/latex/contrib/booktabs/
% Regel von Typesettern: BENUTZE NIE SENKRECHTE STRICHE IN TABELLEN
\usepackage{tikz}
\usetikzlibrary{positioning}
\bibliographystyle{unsrt}
