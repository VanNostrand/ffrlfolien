%ohne [handout] kommen die Overlays
%\documentclass[handout]{beamer}
\documentclass{beamer}
\usepackage{ngerman,graphicx,framed}	%deutsche Sprache, Bilder, Rahmen
\usepackage[utf8]{inputenc}		%direkt deutsche Umlaute und EUR-Zeichen eingeben
\usepackage{lmodern}
\usepackage{textcomp}			%das EUR-Zeichen für OT und T1
\usepackage{color}			%Farbpaket
\usepackage{amsmath}			%Mathepaket - was sonst? =)
\usepackage{amsthm}			%theorem-Paket für Beweise, z.B. begin{proof} oder \qed
\usepackage{amsfonts}			%Schrift um z.B. Mathesymbole wie Folgerungspfeile zu schreiben
\usepackage{amssymb}			%mehr Symbole - yes! =)
\usepackage{stmaryrd}			%\lightning für Widerspruch
\usepackage{flafter}			%Gleitobjekte besser positionieren
\usepackage{placeins}			%it defines a \FloatBarrier command beyond which floats may not pass.
					%A package option allows you to declare that floats may not pass a
					%\section command, but you can place \FloatBarriers wherever you choose.
					%URL: http://www.tex.ac.uk/cgi-bin/texfaq2html?label=floats
\usepackage{mathrsfs}			%Schreibschriftbuchstaben für Mathemodus: \mathscr{Buchstabe}
\usepackage[mathcal]{euscript}		%Kaligraphiebuchstaben für Mathemodus: \mathcal{Buchstabe}
\usepackage{enumerate}			%Aufzählungen mit römischen Zahlen usw.
\usepackage[ngerman]{babel}
\usepackage{algorithmic}		%Quellcode einbauen
\usepackage{longtable}
\usepackage{booktabs}			%für besser aussehende Tabellen als das Standardzeug. 
\usepackage{cite}
\usepackage{listing}
% URL: http://www.ctan.org/tex-archive/macros/latex/contrib/booktabs/
% Regel von Typesettern: BENUTZE NIE SENKRECHTE STRICHE IN TABELLEN
\usepackage{tikz}
\usetikzlibrary{positioning}
\bibliographystyle{unsrt}

\usepackage{eurosym}
%Seiten Druckergerecht:
%\usepackage{pgfpages}
%\pgfpagesuselayout{resize to}[a4paper,border shrink=5mm,landscape]

\title{Freifunk}
\subtitle{und das Abmahnwesen}
\author{Freifunk Rheinland e.V.\\(Funkzelle Aachen)}
\date{\today}


\subject{\input{subject}}
\keywords{\input{keywords}}

\setcounter{tocdepth}{1}

%Berkeley, Antibes, Warsaw
\usetheme{EastLansing}
%\setbeamertemplate{navigation symbols}{}
%\usecolortheme{seahorse}
%\usecolortheme{rose}

\date{\today}

%\setcounter{tocdepth}{1}

% \defbeamertemplate*{footline}{infolines theme}{%
%  \hspace*{2ex}\raisebox{1.5ex}[-1.5ex]{%
%  \color{gray}\tiny\insertframenumber{}/\inserttotalframenumber}\color{black}%
% }% footline

\begin{document}
\frame[plain]{\titlepage}

\frame{
\frametitle{Vortragsinhalt}
\framesubtitle{Gesamtdauer etwa 60 Minuten}
  \tableofcontents
}

\frame{
\frametitle{Hinweise}
\begin{block}{Hintergrund und Warnung bzgl. Vortrag/Folien}
\begin{itemize}
 \item Stand: Deutsches Recht 2012
 \item Technische Umstände: Freifunk Rheinland Firmware 2012.
 \item Folien wurden von juristischem Laien aus Erinnerungen an ein Gespräch über eine Abmahnung mit einem Fachanwalt (IT-Recht, Abmahnungsrecht, \dots) verfasst.
 \item Vortrag und Folien stellen keine juristische, individuelle Beratung dar!
\end{itemize}
\end{block}
\pause
\begin{block}{Allgemeingültiger Rat}
\centering
\begin{framed}
\color{red}\textbf{Beratung eines Fachanwalts/Juristen IT-Recht/Abmahnrecht einholen.}
\end{framed}
\end{block}
}

\section{Inhalt von Abmahnungen}
\frame{
\frametitle{Inhalt von Abmahnungen}
\begin{block}{Abmahnungsschreiben}
\begin{itemize}
 \item Detaillierter Grund (z.B. Urheberrecht, Werk, Datum etc).
 \item Viel einschüchternder Text über Streitwert (10.000 \euro, \dots), Strafzahlungen, Geld, Geld, Geld!1!! [tl;dr]
 \item Angebot außergerichtlicher Einigung mit Frist (paar hundert Euro).
 \item Angehängte Unterlassungserklärung (\textit{impliziert Schuld}), mit Frist.
 \item Vollmacht des Auftraggebers/Urheberrechtsinhabers.
\end{itemize}
\end{block}
}

\section{Verhalten bei Abmahnungen}
\frame{
\frametitle{Verhalten bei Abmahnungen}
\begin{block}{Empfehlung}
\begin{itemize}
 \item Keine Panik, nicht überstürzt handeln! (schwer)
 \item Internettips und Pressemitteilungen mitunter falsch!
 \item Rechtzeitig \textbf{eigenen Anwalt nehmen!}\\Bei sehr knappen Fristen ($\leq2$ Tage): Notfalltelefonnummer.
\end{itemize}
\end{block}
\pause
\begin{block}{Bei Unschuld}
\begin{itemize}
 \item Vom Anwalt erstellte, \textit{modifizierte} Unterlassungserklärung abschicken (Einwurfeinschreiben)
 \item Kopie und Postquittung sammeln, mindestens 30 Jahre!
 \item Ursache beheben: Bei erneutem Vorfall Strafzahlung.
 \item Entweder
 \begin{enumerate}
  \item Klage abwarten (Glücksspiel!)
  \item Preis von Anwalt aushandeln lassen (etwa 50\%)
 \end{enumerate}
\end{itemize}
\end{block}
}

\section{Abmahnen statt Klagen?}
\frame{
\frametitle{Abmahnen statt Klagen?}
\begin{block}{Warum abmahnen, statt sofort klagen?}
\begin{itemize}
 \item Etwa 400 \euro{} pro Anklage vorzahlen (Gericht, eigene Anwälte).
 \item Für \textit{Massen}abmahnung summiert sich das.
 \item Umfang der Schuld/Entschädigung nicht bekannt.
 \item Zahlungsfähigkeit des Angeklagten nicht bekannt.
 \item Kanzleien verhalten sich unterschiedlich.
\end{itemize}
\end{block}
 \pause
\begin{block}{Folgerung}
 Bei kleinen Auftraggebern seltener, aber \textit{nicht ausgeschlossen}.
\end{block}
}

\section{Situation vor Gericht}
\frame{
\frametitle{Situation vor Gericht}
\begin{block}{Urheberrechtsklage}
\begin{itemize}
 \item Gerichtsstand fliegend (\textit{Hamburg}, Berlin, München, \dots)
 \item Anordnung \textit{persönlicher} Anwesenheit (Zeit, Reisekosten, +Anwalt).
 \item Mit 100-$\varepsilon$\% Wahrscheinlichkeit verlieren:
 \pause
\begin{itemize}[<+-|alert@+>]
 \item Schnüffler zertifiziert und beglaubigt.
 \item Fehlerquote nach Gutachten: \textperthousand{}.
 \item Fehler in Anklagegrundlage muss \textit{Angeklagter beweisen}.
 \item Sonstige Gegenargumente: \glqq Schutzbehauptung\grqq $\Rightarrow$ keine Wirkung.
 \item Gerichte überlastet, Textbausteinurteile.
 \item Störerhaftung: Mindestens Teilschuld.
 \item Richter sind niemandem verpflichtet, auch keinen Präzedenzfällen, BGH-Urteilen etc. (Unabhängigkeit nach Gewaltenteilung)
\end{itemize}
\end{itemize}
\end{block}
}

\section{Häufige Vorschläge -- Warum sie ungeeignet sind.}
\frame{
\frametitle{Häufige Vorschläge}
\framesubtitle{Warum sie ungeeignet sind.}
\begin{block}{1/2}
\begin{description}[<+-|alert@+>]
 \item[$\bullet$] Modifizierte Unterlassung aus dem Inet abschicken.
 \item[$\circ$] Kann inhaltlich formell falsch sein, impliziert außerdem Klageabwartung.
 \item[$\bullet$] Vereinsanwalt fragen, pro bono.
 \item[$\circ$] Name? Standort? Pro bono auch für Preisverhandlung oder Gericht und bei vielen Abmahnungen?
 \item[$\bullet$] Kriegskasse des Vereins.
 \item[$\circ$] Kapazität? Auch bei eventueller Nichtmitgliedschaft?
 \item[$\bullet$] Spenden sammeln.
 \item[$\circ$] Keine Gewissheit, auch nur ein Verfahren abzudecken, geschweige mehr Instanzen. Aufwand?
\end{description}
\end{block}
}

\frame{
\frametitle{Häufige Vorschläge}
\begin{block}{2/2}
\begin{description}[<+-|alert@+>]
 \item[$\bullet$] Präzedenzfall schaffen.
 \item[$\circ$] Richter sind unabhängig. Folgeklagen müssen ggf. durch Instanzen und haben keine Garantien.
 \item[$\bullet$] Der Verein wurde dafür gegründet (den Kopf hinzuhalten).
 \item[$\circ$] Der Verein ist niemals betroffen, solange der Anschluss einer Privatperson gehört. Mindestens die Störerhaftung bleibt.
 \item[$\bullet$] Firmware besitzt Filter.
 \item[$\circ$] Nicht so wirksam/up-to-date/leistungsfähig wie Systeme mit kommerziellem Support (IPOQUE), aber auch diese sind umgehbar, wie jedes menschengeschaffene System.
 \item[$\bullet$] BGH 2010: Nur Unterlassung und maximal 100 \euro.
 \item[$\circ$] Diese Pressemeldung ist \textit{falsch}, sie weicht zentral vom wirklichen Urteil ab, wird aber überall so verbreitet.
\end{description}
\end{block}
} 

\section{Lösungsideen zu aktueller Gesetzeslage}
\frame{
\frametitle{Lösungsideen zu aktueller Gesetzeslage}
\begin{block}{}
\begin{itemize}[<+-|alert@+>]
 \item Gesetze ändern
 \begin{itemize}
  \item Kosten und Risiken für Verbraucher senken.
  \item Mindestens die Störerhaftung bearbeiten.
  \item Abmahnwesen betrachten.
  \item Gerichte entlasten!
  \item Erfolg? Große Parteien: Medienlobby. Kleine Parteien: wenig Einfluss.
 \end{itemize}
 \item Nichts darf zum Anschluss des Freifunkers führen.
 \begin{itemize}
  \item Vereinsproxy/-VPN bisher am besten, weil dann der Verein als juristische Person angegriffen wird. Die technische Umsetzung fehlt aber. Im Worstcase trifft es \textit{trotzdem} das Vereinsmitglied, über dessen Anschluss es geschehen ist (Logauswertung des Proxy).
  \item Auslands-VPN: keine echte Garantie, dass die keine Verbindungsdaten haben und herausgeben: Juristischer Druck+Wandel? Leaks? Bestechung bei VPN in Bananenrepublik? \dots?
 \end{itemize}
\end{itemize}
\end{block}
}

\frame{
\frametitle{Weitere Lösungen}
\begin{block}{}
\begin{itemize}
 \item Nur Mitgliedern Zugang erlauben, sie verpflichten sich, keinen Unfug anzustellen und werden zentral authentifiziert (VPN, LDAP, Radius). Weicht je nach Auslegung von Freifunkidee ab.
 \item Geschäftsmodell muss bedroht sein. Verein müsste 1 Cent für die Nutzung nehmen, ist aber gemeinnützig. Weicht prinzipiell auch von der Freifunkidee ab.
 \item Schlechteste: Alles so belassen, weiter auf einen \glqq Präzedenzfall\grqq{} (auf wessen Rücken?) warten.
\end{itemize}
\end{block}
}

\section{Vortragsende, Zusammenfassung}
\frame{
\frametitle{Vortragsende, Zusammenfassung}
\framesubtitle{Vielen Dank fürs Zuhören!}
\begin{block}{Zusammenfassung}
\begin{itemize}
 \item Ziel von Abmahnungen: Geld, nicht Recht.
 \item Betroffen ist z.Z. nicht der Verein, sondern der Anschlussinhaber.
 \item Keinen Internettips/Presseberichten folgen!
 \item Umsonst kommt man nur durch \textit{Glücksspiel} heraus.
 \item Häufig zitierte Vorschläge sind unpraktikabel.
 \item Lösungen sind semi-befriedigend.
\end{itemize}
\end{block}

\begin{block}{Fragen, Diskussion?}
\end{block}

\begin{block}{Folien:}
\centering
\url{https://github.com/VanNostrand}
\end{block}

}

\end{document}